\documentclass{article}
\usepackage{blindtext}
\usepackage{graphicx}
\graphicspath{ {./image/} }

\title{Regula-Falsi Method}
\author{Dennis Mwendwa}
\date{March 2023}

\begin{document}
\maketitle
\section*{Regula-Falsi Method}

Regula-Falsi method or the method of false position is a numerical method for estimating roots of a polynomial. It is a combination of the secant method and bisection methods.

The idea is that if you have a smooth function that doesn’t change much, you can approximate the function with a line using two endpoints $[a, b]$. The endpoints are joined with a chord; The point where the chord crosses the x-axis is the new “guess” for the root. The appropriate endpoint is updated with the new guess, then the algorithm continues, honing in on the actual root.

The Regula-Falsi tends to be faster than the bisection method and requires fewer iterations, this comes with a higher computational cost. Compared to the secant method, it requires more iterations

Algorithm for the Regula–Falsi Method: Given a continuous function f(x)
  \begin{enumerate}
     \item Choose two initial guesses, $a$ and $b$, such that $f(a)$ and $f(b)$ have opposite signs.
     \item Compute the point $c$ where the chord connecting $(a, f(a))$ and $(b, f(b))$ intersects the x-axis. Now, the equation of the chord joining A[a, f(a)] and B[b, f(b)] is given by: $$c = \frac{a.f(b)-b.f(a)}{f(b)-f(a)} $$ 
     \item Compute the value of the function at $c$, $f(c)$.
     \item  Is $f(c) = 0$? If so, stop here. You’ve found the root. If not, continue to the next step.
     \item Replace a or b with c, depending on the sign of f(c) in that if:
     \begin{itemize} 
        \item $f(a). f(b) < 0$ then $b = c$
         \item $f(a). f(b) > 0$ then $a = c$
      \end{itemize}
       \item Continue the process repeatedly until you reach 0 or the desired accuracy.
  \end{enumerate}
  
  \pagebreak
  
 Geometrical representation of the roots of the equation f(x) = 0 can be shown as:
 
 \includegraphics[scale=0.5]{false-position-method}
 
 \section*{Sample Problem}
 Show that $f(x) = x^3 + 3x - 5$ has a root in $[1,2]$, and use the Regula-Falsi Method to determine an approximation to the root that is accurate to at least within $10^{-6}$
 
 Now, the information required to perform the Regula Falsi Method is as follows:
\begin{itemize} 
    \item $f(x) = x^3 + 3x - 5$,
    \item Lower Guess a = 1,
    \item Upper Guess b = 2,
    \item And tolerance $e = 10^{-6}$
\end{itemize}

\textbf{Solution:} 
\begin{itemize}
   \item \emph{Iteration 1}
   
      $a=1$, $b=2$
      
      $f(1)= -1$ , $f(2)=9$
      
      $f(a).f(b) = -1 * 9 = -9 < 0$
      
      $c = \frac{a.f(b)-b.f(a)}{f(b)-f(a)} $ = $1.1$
      
      $f(a).f(c) = f(1).f(1.1) = 0.369 > 0$
      
      Since the above condition is not satisfied, we make c as our new lower  guess i.e. 
$a = c$,
$a = 1.1$
So, we have reduced the interval to :
$[1, 2] -> [1.1, 2]$

Now we check the loop condition i.e.

 fabs$(f(c)) > e$
$f(c) = f(1.1) = -0.369$
fabs$(f(c)) = 0.369 > e = 10^{-6}$
The loop condition is true so we will perform the next iteration.


\item \emph{Iteration 2}

$a = 1.1$, $b = 2$

$f(a)=f(1)=-5$,  $f(b)=f(2)=9$

$c = \frac{a.f(b)-b.f(a)}{f(b)-f(a)} $ = $1.135446686$

$f(a).f(c) = f(1).f(1.135446686) = -0.1297975921 < 0 $   

 Since the above condition is not satisfied, we make $c$ as our new lower guess i.e.
$a = c$,
$a = 1.135446686$

Again we have reduced the interval to :
$[1,2] -> [1.135446686,2]$  

Now we check the loop condition i.e.

 fabs$(f(c)) > e$
$f(c) = -0.1297975921$

fabs$(f(c)) = 0.1297975921 > e = 10^{-6} $
The loop condition is true so we will perform the next iteration.

As you can see, it converges to a solution which depends on the tolerance and number of iteration the algorithm performs.
\end{itemize}

\pagebreak

 \section*{Python Implementation}
This program implements false position (Regula Falsi) method for finding real root of nonlinear equation in python programming language.

\begin{verbatim}
import math
import matplotlib.pyplot as plt
import numpy as np
import timeit

x = np.array(range(-6,6))
y = x**3 + 3*x - 5
plt.plot(x,y)
plt.grid()
plt.show()

def f(x):
    return x**3+3*x-5


def regula(a,b):
    x = 0
    i=1
    condition = True
    while condition:
        n = str(x)
        x = a - ((b-a)/(f(b)-f(a)))*f(a)
        if f(x)<0:
            a=x
        else:
            b=x
        print("Iteration number: ", i, "     x = ",x, "     f(x) = ",f(x))

        m = str(x)
        if m[0:t+3]==n[0:t+3]:
            condition = False
        else:
            condition = True
            i = i+1
    
    print("Root found at x = ", x)
    print("Time taken: ",timeit.timeit())


a = input("First approximation root: ")
b = input("Second approximation root: ")
t = input("Enter precision of decimal places: ")
a = float(a)
b = float(b)
t = int(t)

if f(a)*f(b)>0:
    print("Given appproximation roots do not give a solution")
    print("Try again with different values")
else:
    regula(a,b)
\end{verbatim}

The output:
\begin{verbatim}
Iteration number:  1      x =  1.1      f(x) =  -0.3689999999999998
Iteration number:  2      x =  1.1354466858789627      f(x) =  -0.1297975921309309
Iteration number:  3      x =  1.147737970248558       f(x) =  -0.04486805098132063
Iteration number:  4      x =  1.1519657086726893      f(x) =  -0.015415586390996161
Iteration number:  5      x =  1.1534157744799958      f(x) =  -0.005285298529249971
Iteration number:  6      x =  1.1539126438421201      f(x) =  -0.0018107788348853404
Iteration number:  7      x =  1.1540828403853087      f(x) =  -0.0006202314857430835
Iteration number:  8      x =  1.154141132418883       f(x) =  -0.00021242488214312516
Iteration number:  9      x =  1.1541610965554805      f(x) =  -7.275190177225e-05
Iteration number:  10     x =  1.154167933876746       f(x) =  -2.491603862431191e-05
Iteration number:  11     x =  1.1541702755129777      f(x) =  -8.533204644223247e-06
Iteration number:  12     x =  1.15417107747201        f(x) =  -2.922434727103962e-06
Iteration number:  13     x =  1.1541713521252337      f(x) =  -1.000869155554085e-06
Iteration number:  14     x =  1.1541714461878683      f(x) =  -3.427754666773808e-07
Iteration number:  15     x =  1.1541714784022312      f(x) =  -1.1739298244606289e-07

Root found at x =  1.1541714784022312
Time taken:  0.034652400005143136
\end{verbatim}

\pagebreak
\section*{Limitations}
While Regula Falsi Method, like Bisection Method, is always convergent; meaning that it is always leading towards a specific limit and relatively simple to understand,  there are also some drawbacks when this algorithm is used. As both Regula-Falsi and Bisection method are similar, there are some common limitaions both the algorithms have i.e. :
\begin{itemize}
\item \textbf{Rate of convergence:} 
The convergence of the regula falsi method can be very slow in some cases(May converge slowly for functions with big curvatures) as explained above.
\item \textbf{Relies on sign changes:} If a function $f (x)$ is such that it just touches the x -axis for example say $f(x) = 2$ then it will not be able to find lower guess $(a)$ such that $f(a)*f(b) < 0$
\item \textbf{Cannot detect Multiple Roots:}Like Bisection method, Regula Falsi Method fails to identify multiple different roots, which makes it less desirable to use compared to other methods that can identify multiple roots.
\end{itemize}





\section*{References}
\begin{enumerate}
\item Stephanie Glen. \emph{"Regula-Falsi method: Definition, Example"} From StatisticsHowTo.com:
\item  Regula Falsi Method for finding root of a polynomial-\emph{Eklavya Chopra}
\item Byju's Learning-\emph{False Position Method}
\item Youtube
\end{enumerate}
\end{document}